\chapter{Spaces}

\section{Types of space}

\definition{Topological space}{  An ordered pair $(X, \tau)$, where $X$ is a set
and $\tau$ is a collection of subsets of $X$, satisfying the following
properties:
\begin{enumerate}
\item The empty set and $X$ belong to $\tau$
\item Any union of members of $\tau$ belongs to $\tau$
\item The intersection of any finite number of members of $\tau$ belongs to $\tau$.
\end{enumerate}
The elements of $\tau$ are called \textit{open sets} and $\tau$ is called a
\textit{topology} in $X$.}{A topology is useful to define concepts such as
continuity of functions between topological spaces.}

\definition{Metric space}{
  A set together with a metric defining a distance between any two points in the
  set which satisfies:
  \begin{enumerate}
  \item identity of indiscernibles: $d(x, y) = 0 \iff x = y$
  \item symmetry: $d(x, y) = d(y, x)$
  \item triangle inequality: $d(x, z) \leq d(x, y) + d(y, z)$
  \end{enumerate}
}{Positivity of $d(x, y)$ for all $x \neq y$ follows from these conditions. A
  metric space is, in some sense, less general than a topological space, since
  all metric spaces give rise to a topological space (not sure if the converse
  is true).}

\section{Properties of spaces}

\subsection{Properties of topological spaces}

\definition{Separable space}{A topological space is separable iff. there exists
  a countable sequence $\{x_n\}_{n=1}^\infty$ of elements in the space such that
  every non-empty open subset of the space contains at least one element of the
  sequence. }{Open subsets are required as in some sense, it is not possible to
  make them as small as closed subsets: \eg every non-empty open subset of
  $\real$ is uncountable whereas a closed subset can have a single member.}

\definition{Completely metrizable space}{ A topoological space (X, T) for which
  there exists a metric d such that (X, d) is a complete metric space and which
  induces the topology T. }{}

\subsection{Properties of metric spaces}

\definition{Complete metric space}{ A metric space M is complete if every
  Cauchy sequence in M has a limit in M. A Cauchy sequence is a sequence $x_1,
  x_2, \ldots$ such that, for any $\epsilon > 0$, there exists an $N$ such
  that for any $m, n > N$, $d(x_m, x_n) < \epsilon$. }{Intuitively, a space
  with `no points missing'. For example, the real numbers are a complete
  metric space while the rational numbers are incomplete.}

\section{Properties of points/sets}

\subsection{In metric spaces}
At least some of the following have more general definitions which apply for
topological spaces. Here are the metric-space versions for simplicity.

\definition{Interior point}{A point $x$ is an interior point of $S$ ($S$ being a
  subset of a metric space) if there exists some $\epsilon > 0$ such that $d(x,
  y) < \epsilon \implies y \in S$.}{}

\definition{Open set}{}{}

\definition{Closed set}{}{The properties of begin open/closed are not mutually
  exclusive.}
