\chapter{Topological spaces}

% Hello.

% \begin{adjustwidth}{1cm}{}
%   Hello.
% \end{adjustwidth}

% Hello.

% \begin{adjustwidth}{1cm}{}
%   Hello.
% \end{adjustwidth}

% \begin{adjustwidth}{1cm}{}
%   Hello.
% \end{adjustwidth}

\definition{Separable space}{ A space is separable iff. there exists a countable
  sequence $\{x_n\}_{n=1}^\infty$ of elements in the space such that every
  non-empty open subset of the space contains at least one element of the
  sequence. }{Open subsets are required as in some sense, it is not possible to
  make them as small as closed subsets: \eg every non-empty open subset of \real
  is uncountable whereas a closed subset can have a single member.}

\definition{Metric space}{
  A set together with a metric defining a distance between any two points in the
  set which satisfies:
  \begin{enumerate}
  \item identity of indiscernibles: $d(x, y) = 0 \iff x = y$
  \item symmetry: $d(x, y) = d(y, x)$
  \item triangle inequality: $d(x, z) \leq d(x, y) + d(y, z)$
  \end{enumerate}
  }{Positivity of $d(x, y)$ for all $x \neq y$ follows from these conditions.}

\definition{Complete metric space}{
  A metric space M is complete if every Cauchy sequence in M has a limit in M. A
  Cauchy sequence is a sequence $x_1, x_2, \ldots$ such that, for any $\epsilon
  > 0$, there exists an $N$ such that for any $m, n > N$, $d(x_m, x_n) <
  \epsilon$.
  }{For example, the real numbers are a complete metric space while the rational
    numbers are incomplete.}